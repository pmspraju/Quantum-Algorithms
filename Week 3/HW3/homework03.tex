\documentclass[oneside]{amsart}  % {{{1
\usepackage{amssymb, mathtools, enumitem, booktabs, float, caption, tikz}
\usepackage[final]{hyperref, microtype}

% structure symbols
\newcommand{\NN}{\mathbb{N}}
\newcommand{\ZZ}{\mathbb{Z}}
\newcommand{\QQ}{\mathbb{Q}}
\newcommand{\RR}{\mathbb{R}}
\newcommand{\CC}{\mathbb{C}}
\renewcommand{\AA}{\mathbb{A}}
\newcommand{\V}{\mathcal{V}}
\newcommand{\m}[1]{\mathbb{#1}}      % models
\newcommand{\alg}[1]{\m{#1}}         % algebras
\newcommand{\rel}[1]{\mathbf{#1}}    % relational structures
\newcommand{\cl}[1]{\mathcal{#1}}    % classes
\newcommand{\Comp}[1]{\textsf{#1}}   % complexity classes

% theorems and similar environments
\theoremstyle{plain}
  \newtheorem{thm}{Theorem}           \newtheorem*{thm*}{Theorem}
  \newtheorem{claim}{Claim}[thm]      \newtheorem*{claim*}{Claim}
  \newtheorem{conj}[thm]{Conjecture}  \newtheorem*{conj*}{Conjecture}
  \newtheorem{cor}[thm]{Corollary}    \newtheorem*{cor*}{Corollary}
  \newtheorem{lem}[thm]{Lemma}        \newtheorem*{lem*}{Lemma}
  \newtheorem{prop}[thm]{Proposition} \newtheorem*{prop*}{Proposition}
\theoremstyle{definition}
  \newtheorem{defn}[thm]{Definition} \newtheorem*{defn*}{Definition}
  \newtheorem{ex}[thm]{Example}      \newtheorem*{ex*}{Example}
\theoremstyle{remark}
  \newtheorem{rk}[thm]{Remark}  \newtheorem*{rk*}{Remark}
\newcommand{\Case}[1]{\smallskip \textbf{Case #1:}}
\newenvironment{claimproof}
  {
    \begin{proof}[Proof of claim]
    \renewcommand{\qedsymbol}{\ensuremath{\circ}}
  } {
    \end{proof}
  }

% custom commands
\DeclareMathOperator{\Cg}{Cg}
\DeclareMathOperator{\Clo}{Clo}
\DeclareMathOperator{\Con}{Con}
\DeclareMathOperator{\Inv}{Rel}
\DeclareMathOperator{\Pol}{Pol}
\DeclareMathOperator{\Sg}{Sg}
\DeclareMathOperator{\diag}{diag}
\newcommand{\bmat}[1]{ \begin{bmatrix} #1 \end{bmatrix} }
\newcommand{\Bmat}[1]{ \begin{Bmatrix} #1 \end{Bmatrix} }
\newcommand{\pmat}[1]{ \begin{pmatrix} #1 \end{pmatrix} }
\newcommand{\mat}[1]{ \begin{matrix} #1 \end{matrix} }
\newcommand{\ds}[1]{ \displaystyle{#1} }
\newcommand{\stack}[2]{\genfrac{}{}{0pt}{}{#1}{#2}}
\newcommand{\Proofitem}[1]{\smallskip \textbf{#1:}}

% tikz
\usetikzlibrary{calc, matrix, positioning}
\newcommand{\TikzCenterNode}{\node (center) at ($(current bounding box.north east)!0.5!(current bounding box.south west)$) {\phantom{$\cdot$}};}

% misc
\numberwithin{equation}{section}  % number equations within sections
\renewcommand{\epsilon}{\varepsilon}
\renewcommand{\phi}{\varphi}
\newcommand{\todo}[1]{\noindent\textbf{TODO: #1}}
\renewcommand*{\arraystretch}{1.25} 
\setlist[enumerate]{itemsep=0.5em}
\setlist[itemize]{itemsep=0.5em}

% homework/exam specific
\usepackage[margin=1.5in, top=1in, bottom=1in]{geometry}
\usepackage{environ}
\newlist{questions}{enumerate}{2}
\setlist[questions]{wide=0pt, parsep=0.5em, listparindent=\parindent, font=\bfseries}
\setlist[questions,1]{label=\arabic*., itemsep=2\parsep, topsep=\itemsep+\parsep}
\setlist[questions,2]{labelindent=\parindent, label=(\roman*), itemsep=\parsep}
\setlist[questions,3]{labelindent=2\parindent, label=(\alph*)}
\newcommand{\SolnItem}[1]{\smallskip\indent\textrm{\bf#1:}}
\newenvironment{proofpart}[1] { \begin{proof}[\SolnItem{#1} Proof.] } { \end{proof} }
\newif\ifsolutions
\NewEnviron{solution}
  {
    \ifsolutions
      \bigskip \noindent \textbf{Solution:} \BODY 
    \fi
  }
  { }

% document specific stuff
% quantum
\DeclareMathOperator{\Cspan}{ \CC-span }
\DeclareMathOperator{\Tr}{ Tr }
\newcommand{\B}{ \mathfrak{B} }
\newcommand{\bra}[1]{ \left< #1 \right| }
\newcommand{\ket}[1]{ \left| #1 \right> }
\newcommand{\bracket}[2]{ \left< #1 \mid #2 \right> }
\tikzset{gate-c/.style = {draw, circle, inner sep=0.25em}}
\tikzset{gate-r/.style = {draw, rounded corners, rectangle}}
\tikzset{control/.style = {draw, fill, circle, inner sep=0.1em}}
\tikzset{circuit/.style = {matrix of math nodes, cells={anchor=center}, column 1/.style={anchor=east}}}
\newcommand{\tikzMeasure}{
  \begin{tikzpicture}[scale=0.2, baseline=0pt]
    \draw (0,0) arc (180-20:20:1);
    \draw[->] (current bounding box.north|-current bounding box.south) -- ++(60:1.5);
  \end{tikzpicture} }

% computation
\renewcommand{\O}{\mathcal{O}}
\DeclareMathOperator{\poly}{poly}
%----------------------------------------------------------------------------}}}1

\begin{document}  %\solutionstrue
% title matter {{{1
\title[Quantum Algorithms, HW 3 Additional Problems]
  {Quantum Algorithms \\ Homework 3 Additional Problems}
\author{Prof.~Matthew Moore}
\maketitle

\vspace{-0.7em} \begin{center}
  \sc Due: 2021-02-23
\end{center}
%----------------------------------------------------------------------------}}}1

\vspace{2em}

\begin{questions}
% Fermat's theorem   {{{1
% ID: eiNoh6oo
\item Compute $\ds{2^{3^{4^5}} \bmod 79}$. I suggest that you do this without
using a computer. 
\\ \noindent 
[\emph{Hint: $78 = 2\cdot 3\cdot 13$.}]
%----------------------------------------------------------------------------}}}1
% Fermat's theorem   {{{1
% ID: co0iu
\item \begin{questions}
  \item Compute $14^{15} \bmod{10}$. [\emph{Hint: look for a pattern
    or use $10 = 2\cdot 5$.}]
\begin{solution}  % {{{2
Since $14\equiv 4 \pmod{10}$, we have that $14^{15} \equiv 4^{15}
\pmod{10}$. A table of powers of $4$ modulo $10$ is given below.
\[ \begin{array}{l|cccccccc}
  k             & 1 & 2 & 3 & 4 & \cdots & 2\ell & 2\ell + 1 & \cdots \\ \midrule
  4^k \bmod{10} & 4 & 6 & 4 & 6 & \cdots & 6     & 4         & \cdots
\end{array} \]
From it, we can see that $4^k \equiv 4 \pmod{10}$ if $k$ is odd and $4^k
\equiv 8\pmod{10}$ is $k$ is even. Since $15$ is odd, we have that
\[
  14^{15}
  \equiv 4^{15}
  \equiv 4 \pmod{10}.
\]

Proceeding the other way suggested using the hint, we consider
\[
  14^{15} \bmod{2}
  \qquad\text{and}\qquad
  14^{15} \bmod{5}.
\]
Since $14 \equiv 0 \pmod{2}$, we have
\[
  14^{15}
  \equiv 0^{15}
  \equiv 0 \pmod{2}.
\]
Since $14 \equiv -1 \pmod{5}$, we have
\[
  14^{15}
  \equiv (-1)^{15}
  \equiv -1
  \equiv 4 \pmod{5}
\]
The first of these tells us that $14^{15} = 2k$, and substituting this
into the second one tells us that $2k \equiv -1 \pmod{5}$, so $k \equiv
2 \pmod{5}$. Therefore $k = 2+5\ell$, and so
\[
  14^{15}
  = 2k
  = 2\cdot (2 + 5\ell)
  = 4 + 10 \ell.
\]
It follows that $14^{15} \equiv 4 \pmod{10}$, just as we saw above.
\end{solution}   % }}}2
  \item Compute $\ds{13^{14^{15}} } \bmod{11}$.
\begin{solution}  % {{{2
The previous part implies that $14^{15} = 4 + 10\beta$. Since $13$ and
$11$ are coprime, Fermat's theorem applies. Using it and the previous
part, we have
\[
  13^{14^{15}}
  \equiv
  13^{4+10\beta}
  \equiv
  13^4 \cdot \big( 13^{10} \big)^{\beta}
  \stackrel{\text{Fermat}}{\equiv} 13^4 \cdot (1)^{\beta}
  \equiv 13^4
  \equiv 2^4
  \equiv 16
  \equiv 5 \pmod{11}.
\]
\end{solution}   % }}}2
\end{questions}
%---------------------------------------------------------------------------}}}1
% python, Miller-Rabin test   {{{1
% ID: uoK9Uh3m
\item Implement the Miller-Rabin probabilistic primality testing algorithm as
presented in class (or in the textbook). Fill in the function
\verb|is_prime_MR(q)| in the python source file. You need only submit your
function with the homework, not the entire source file.
%----------------------------------------------------------------------------}}}1
% Miller-Rabin test   {{{1
% ID: laiZool1
\item Find five pairs of numbers $q\in \ZZ$ and $a \in \{1, \dots, q-1\}$ such
that $q$ is composite but passes the Miller-Rabin test with the given choice of
$a$.
%----------------------------------------------------------------------------}}}1
\end{questions} \end{document}

\vfill \begin{center} \textbf{Problems continue on the next page.} \end{center} \vfill
