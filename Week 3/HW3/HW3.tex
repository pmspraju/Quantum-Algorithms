\documentclass [12pt]{article}
\setlength{\parindent}{0em}
\setlength{\parskip}{0.25in}
\usepackage{geometry}
\geometry{verbose,letterpaper,tmargin=0.5in,bmargin=1.0in,lmargin=.70in,rmargin=.70in}
\usepackage{graphicx}
\usepackage{amsmath}
\usepackage{amssymb}
\usepackage{amsthm}
\theoremstyle{definition}
\newtheorem{exmp}{Example}[section]
\usepackage{tikz}
\usetikzlibrary{arrows,decorations.pathmorphing,backgrounds,positioning,fit,petri,calc,matrix}
\usepackage{slashbox}
\usepackage{listings}

\definecolor{dkgreen}{rgb}{0,0.6,0}
\definecolor{gray}{rgb}{0.5,0.5,0.5}
\definecolor{mauve}{rgb}{0.58,0,0.82}

\lstset{frame=tb,
  language=Python,
  aboveskip=3mm,
  belowskip=3mm,
  showstringspaces=false,
  columns=flexible,
  basicstyle={\small\ttfamily},
  numbers=none,
  numberstyle=\tiny\color{gray},
  keywordstyle=\color{blue},
  commentstyle=\color{dkgreen},
  stringstyle=\color{mauve},
  breaklines=true,
  breakatwhitespace=true,
  tabsize=3
}


\title{Home Work 3}
\author{Madhu Peduri}
\date{01/13/2021}

\begin{document}
\section*{Homework 3}

{\bf 1.} Implement the Miller-Rabin probabalistic testing algorithm as presented in class (or in the text book). \\

\begin{lstlisting}
def is_prime_MR(q): # {{{
  # Fill in each step (as outlined in class or the textbook) below the relevant
  # comment.
  #
  # At Step 3, you will need to produce a random element of {1, ..., q-1}. Use
  # randrange() to do this: a = randrange(1, q). Note that modulus in python is
  # positive, so -1 should be represented as q-1.

  if q <= 1:
    return False
    
  # Step 1
  if ((q % 2) == 0):
    return False

  # Step 2
  l = q - 1
  k = 0
  while((l % 2) == 0):
    k += 1
    l = l // 2
   
  # Step 3
  a = randrange(2, (q - 1))
  a = 2
  # Step 4
  p = [2**val for val in range(k+1)]
  al = a**l
  allist = [((al**val) % q) for val in p]
  print(allist)
  
  # Test 1
  if(allist[-1] != 1):
        return False   

  # Test 2
  ind_1 = [i for i, e in enumerate(allist) if e == 1]
  if(len(ind_1) > 0):
    for val in ind_1:
        if(val > 0):
            if(allist[val-1] in [1,-1]):
                return False
\end{lstlisting}

\newpage

{\bf 2.} Find five pairs of numbers $q \in Z $ and $a \in \{1, \dots, q-1\}$ such that $q$ is composite but passes the Miller-Rabin test with the given choice of $a$.

\phantom{1em} Below are the numbers and bases of Composite numbers which are categorized as primes by Miller-Rabin test. Also known as Pseudoprimes. Let $n$ be the number and $a$ be the base. 

\phantom{1em} {\bf 1.} $n = 21$ for base $a = 20$. $21 = 3*7$ is a composite. \\
\phantom{1000em} Test 1: $20^{20} \equiv 1 (mod 21)$. According to MR primality, this is not composite.\\


\phantom{1em} {\bf 2.} $n = 25$ for base $a = 7$. $25 = 5*5$ is a composite. \\
\phantom{1000em} Test 2: $a^{l} = \{18,24,1,1\}$. According to MR primality, this is not composite.

\phantom{1em} {\bf 3.} $n = 25$ for base $a = 18$. $25 = 5*5$ is a composite. \\
\phantom{1000em} Test 2: $a^{l} = \{7, 24, 1, 1\}$. According to MR primality, this is not composite.

\phantom{1em} {\bf 4.} $n = 49$ for base $a = 18$. $49 = 7*7$ is a composite. \\
\phantom{1000em} Test 2: $a^{l} = \{1, 1, 1, 1, 1\}$. According to MR primality, this is not composite.

\phantom{1em} {\bf 5.} $n = 49$ for base $a = 19$. $49 = 7*7$ is a composite. \\
\phantom{1000em} Test 2: $a^{l} = \{48, 1, 1, 1, 1\}$. According to MR primality, this is not composite.


\end{document}