\documentclass[oneside]{amsart}  % {{{1
\usepackage{amssymb, mathtools, enumitem, booktabs, float, caption, tikz}
\usepackage[final]{hyperref, microtype}

% structure symbols
\newcommand{\NN}{\mathbb{N}}
\newcommand{\ZZ}{\mathbb{Z}}
\newcommand{\QQ}{\mathbb{Q}}
\newcommand{\RR}{\mathbb{R}}
\newcommand{\CC}{\mathbb{C}}
\renewcommand{\AA}{\mathbb{A}}
\newcommand{\V}{\mathcal{V}}
\newcommand{\m}[1]{\mathbb{#1}}      % models
\newcommand{\alg}[1]{\m{#1}}         % algebras
\newcommand{\rel}[1]{\mathbf{#1}}    % relational structures
\newcommand{\cl}[1]{\mathcal{#1}}    % classes
\newcommand{\Comp}[1]{\textsf{#1}}   % complexity classes

% theorems and similar environments
\theoremstyle{plain}
  \newtheorem{thm}{Theorem}           \newtheorem*{thm*}{Theorem}
  \newtheorem{claim}{Claim}[thm]      \newtheorem*{claim*}{Claim}
  \newtheorem{conj}[thm]{Conjecture}  \newtheorem*{conj*}{Conjecture}
  \newtheorem{cor}[thm]{Corollary}    \newtheorem*{cor*}{Corollary}
  \newtheorem{lem}[thm]{Lemma}        \newtheorem*{lem*}{Lemma}
  \newtheorem{prop}[thm]{Proposition} \newtheorem*{prop*}{Proposition}
\theoremstyle{definition}
  \newtheorem{defn}[thm]{Definition} \newtheorem*{defn*}{Definition}
  \newtheorem{ex}[thm]{Example}      \newtheorem*{ex*}{Example}
\theoremstyle{remark}
  \newtheorem{rk}[thm]{Remark}  \newtheorem*{rk*}{Remark}
\newcommand{\Case}[1]{\smallskip \textbf{Case #1:}}
\newenvironment{claimproof}
  {
    \begin{proof}[Proof of claim]
    \renewcommand{\qedsymbol}{\ensuremath{\circ}}
  } {
    \end{proof}
  }

% custom commands
\DeclareMathOperator{\Cg}{Cg}
\DeclareMathOperator{\Clo}{Clo}
\DeclareMathOperator{\Con}{Con}
\DeclareMathOperator{\Inv}{Rel}
\DeclareMathOperator{\Pol}{Pol}
\DeclareMathOperator{\Sg}{Sg}
\DeclareMathOperator{\diag}{diag}
\newcommand{\bmat}[1]{ \begin{bmatrix} #1 \end{bmatrix} }
\newcommand{\Bmat}[1]{ \begin{Bmatrix} #1 \end{Bmatrix} }
\newcommand{\pmat}[1]{ \begin{pmatrix} #1 \end{pmatrix} }
\newcommand{\mat}[1]{ \begin{matrix} #1 \end{matrix} }
\newcommand{\ds}[1]{ \displaystyle{#1} }
\newcommand{\stack}[2]{\genfrac{}{}{0pt}{}{#1}{#2}}
\newcommand{\Proofitem}[1]{\smallskip \textbf{#1:}}

% tikz
\usetikzlibrary{calc, matrix, positioning}
\newcommand{\TikzCenterNode}{\node (center) at ($(current bounding box.north east)!0.5!(current bounding box.south west)$) {\phantom{$\cdot$}};}

% misc
\numberwithin{equation}{section}  % number equations within sections
\renewcommand{\epsilon}{\varepsilon}
\renewcommand{\phi}{\varphi}
\newcommand{\todo}[1]{\noindent\textbf{TODO: #1}}
\renewcommand*{\arraystretch}{1.25} 
\setlist[enumerate]{itemsep=0.5em}
\setlist[itemize]{itemsep=0.5em}

% homework/exam specific
\usepackage[margin=1.5in, top=1in, bottom=1in]{geometry}
\usepackage{environ}
\newlist{questions}{enumerate}{3}
\setlist[questions]{wide=0pt, parsep=0.5em, listparindent=\parindent, font=\bfseries}
\setlist[questions,1]{label=\arabic*., itemsep=2\parsep, topsep=\itemsep+\parsep}
\setlist[questions,2]{labelindent=\parindent, label=(\roman*), itemsep=\parsep}
\setlist[questions,3]{labelindent=2\parindent, label=(\alph*)}
\newcommand{\SolnItem}[1]{\smallskip\indent\textrm{\bf#1:}}
\newenvironment{proofpart}[1] { \begin{proof}[\SolnItem{#1} Proof.] } { \end{proof} }
\newif\ifsolutions
\NewEnviron{solution}
  {
    \ifsolutions
      \bigskip \noindent \textbf{Solution:} \BODY 
    \fi
  }
  { }

% document specific stuff
% quantum
\DeclareMathOperator{\Cspan}{ \CC-span }
\DeclareMathOperator{\Tr}{ Tr }
\newcommand{\B}{ \mathfrak{B} }
\newcommand{\bra}[1]{ \left< #1 \right| }
\newcommand{\ket}[1]{ \left| #1 \right> }
\newcommand{\bracket}[2]{ \left< #1 \mid #2 \right> }
\tikzset{gate-c/.style = {draw, circle, inner sep=0.25em}}
\tikzset{gate-r/.style = {draw, rounded corners, rectangle}}
\tikzset{control/.style = {draw, fill, circle, inner sep=0.1em}}
\tikzset{circuit/.style = {matrix of math nodes, cells={anchor=center}, column 1/.style={anchor=east}}}
\newcommand{\tikzMeasure}{
  \begin{tikzpicture}[scale=0.2, baseline=0pt]
    \draw (0,0) arc (180-20:20:1);
    \draw[->] (current bounding box.north|-current bounding box.south) -- ++(60:1.5);
  \end{tikzpicture} }

% computation
%\usepackage[final]{minted}
%\usemintedstyle{bw, mathescape, escapeinside=;;}
\renewcommand{\O}{\mathcal{O}}
\DeclareMathOperator{\poly}{poly}
%----------------------------------------------------------------------------}}}1

\begin{document}  %\solutionstrue
% title matter {{{1
\title[Quantum Algorithms, Final Exam]{Quantum Algorithms \\ Final Exam}
\author{Prof.~Matthew Moore}
\maketitle

\thispagestyle{empty} \setcounter{page}{0}

\vspace{-0.7em} \begin{center}
  \sc Due: 2021-05-11 at 10:00
\end{center}

\vspace{4em}

\noindent \textbf{Name:} \underline{\hspace{20em}}

\vspace{2em}

\noindent \textbf{Instructions} % {{{
\begin{itemize}[itemsep=0.5em, topsep=\itemsep]
  \item You may make use of any non-human assistance --- any book, the web
    (but do not ask for help online), etc. Solutions must be self-contained.

  \item Solutions given with little or no justification may receive little
    or no credit.

  \item Solutions will be graded based on correctness, quality, and
    presentation. Turn in something that you are proud of.

  \item There are 3 problems. \textbf{Submit only 2 for grading.}
\end{itemize}
%----------------------------------------------------------------------------}}}

\vspace{2em}

\noindent ``I pledge on my honor that I have neither given nor received
unauthorized aid on this assignment.''

\vspace{4em}

\noindent \textbf{Signature:} \underline{\hspace{20em}}
%----------------------------------------------------------------------------}}}1

\newpage \phantom{.} \vspace{-2em}

\begin{questions}
% QFT for finite Abelian groups   {{{1
% ID: tae6Ukof
\item Let $\m{G}$ be a finite Abelian group such that
\[
  \m{G}
  \cong \prod_{i=1}^k \ZZ/m_i\ZZ
\]
by the Fundamental Theorem of Finitely Generated Abelian Groups. Regard elements
$g\in G$ as tuples $g = (g_i) \in \prod \ZZ/m_i\ZZ$. Recall that the Quantum
Fourier Transform for $\alg{G}$ was defined
\[
  \cl{F}_{\alg{G}}
  \coloneqq \frac{1}{\sqrt{|\alg{G}|}} \sum_{g,h\in G} \mu(g,h) \ket{g}\bra{h}
\]
where
\[
  \mu(g,h) \coloneqq \prod_{i=1}^k \omega_{m_i}^{g_i h_i},
  \hspace{6em}
  \omega_{m_i} \coloneqq \exp(2\pi i / m_i).
\]
Show that $\ds{ \cl{F}_{\alg{G}} = \bigotimes_{i=1}^k \cl{F}_{\ZZ/m_i\ZZ} }$.
%----------------------------------------------------------------------------}}}1

\newpage \phantom{.} \vspace{-2em}

% Kronecker product, tensors   {{{1
% ID: too3Maev
\item Recall that the Kronecker product of \emph{matrices} $A$ and $B$ is
defined
\[
  A * B
  \coloneqq 
  \pmat{ a_{11} B & \cdots & a_{1n} B \\
         \vdots   &        & \vdots \\
         a_{m1} B & \cdots & a_{mn} B },
  \hspace{6em}
  A = (a_{ij}),
\]
and that given a \emph{linear operator} $C: \alg{T} \to \alg{U}$, we denote the
matrix of $C$ relative to some specified ordered bases for $\alg{T}$ and
$\alg{U}$ as $[C]$.

\smallskip

Let $\alg{V}, \alg{W}, \alg{X}, \alg{Y}$ be vector spaces with respective
ordered bases
\begin{align*}
  \cl{V} &= \big\{ \ket{v_1}, \ket{v_2} \big\},
    & \cl{W} &= \big\{ \ket{w_1}, \ket{w_2}, \ket{w_3} \big\}, \\
  \cl{X} &= \big\{ \ket{x_1}, \ket{x_2}, \ket{x_3} \big\},
    & \cl{Y} &= \big\{ \ket{y_1}, \ket{y_2} \big\}.
\end{align*}
Define linear operators $D: \alg{V}\to \alg{W}$ and $E: \alg{X}\to \alg{Y}$ by
\begin{align*}
  D\ket{v_1} &= \ket{w_1} + \ket{w_2} + \ket{w_3},
    & E\ket{x_1} &= \ket{y_1} - \ket{y_2}, \\
  D\ket{v_2} &= 2\ket{w_2} - \ket{w_3},
    & E\ket{x_2} &= 2\ket{y_2}, \\
  && E\ket{x_3} &= \ket{y_1} + \ket{y_2}.
\end{align*}
Show that $[D]*[E] = [D\otimes E]$, where the bases for $\alg{V}\otimes \alg{X}$
and $\alg{W}\otimes \alg{Y}$ are the usual lexicographically ordered bases. You
may do this by direct calculation if you wish.
%----------------------------------------------------------------------------}}}1

\newpage \phantom{.} \vspace{-2em}

% discrete log problem   {{{1
% ID: ai7ChahJ
\item Given a group $\alg{G}$, recall that the discrete logarithm problem takes
as input elements $a, b\in G$ such that $b^s = a$, and outputs the number $s$.
Recall that the quantum solution to the discrete logarithm problem involves
running the eigenvalue estimation circuit in series using the operators $U_b$
and $U_a^{\dagger}$.

\begin{questions}
  \item Show that the state in the circuit before passing through the two
  inverse Quantum Fourier Transform blocks is proportional to
  \[
    \sum_{x,y\in \{0,1\}^n} \ket{x, y, b^x a^{-y}}.
  \]

  \item Carefully show that the state after passing through the two inverse
  Quantum Fourier Transform blocks but before measurement is proportional to
  \[
    \sum_{\stack{z,w\in \{0,1\}^n}{sw+z=0}} \ket{z,w} 
      \otimes \sum_{k=1}^m \exp(-2\pi i (kw)/2^n) \ket{b^k},
  \]
  where $m$ is the period/order of $b$ in $\alg{G}$. [\emph{Hint: use $b^s =
  a$}.]

  \item Conclude that measuring the top two registers of the circuit produces
  pairs $\ket{u, v}$ such that $b^u a^{-v} = 1$. Explain how to use such pairs
  to find $s$.
\end{questions}
%----------------------------------------------------------------------------}}}1
\end{questions}
\end{document}
