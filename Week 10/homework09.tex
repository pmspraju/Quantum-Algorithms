\documentclass[oneside]{amsart}  % {{{1
\usepackage{amssymb, mathtools, enumitem, booktabs, float, caption, tikz}
\usepackage[final]{hyperref, microtype}

% structure symbols
\newcommand{\NN}{\mathbb{N}}
\newcommand{\ZZ}{\mathbb{Z}}
\newcommand{\QQ}{\mathbb{Q}}
\newcommand{\RR}{\mathbb{R}}
\newcommand{\CC}{\mathbb{C}}
\renewcommand{\AA}{\mathbb{A}}
\newcommand{\V}{\mathcal{V}}
\newcommand{\m}[1]{\mathbb{#1}}      % models
\newcommand{\alg}[1]{\m{#1}}         % algebras
\newcommand{\rel}[1]{\mathbf{#1}}    % relational structures
\newcommand{\cl}[1]{\mathcal{#1}}    % classes
\newcommand{\Comp}[1]{\textsf{#1}}   % complexity classes

% theorems and similar environments
\theoremstyle{plain}
  \newtheorem{thm}{Theorem}           \newtheorem*{thm*}{Theorem}
  \newtheorem{claim}{Claim}[thm]      \newtheorem*{claim*}{Claim}
  \newtheorem{conj}[thm]{Conjecture}  \newtheorem*{conj*}{Conjecture}
  \newtheorem{cor}[thm]{Corollary}    \newtheorem*{cor*}{Corollary}
  \newtheorem{lem}[thm]{Lemma}        \newtheorem*{lem*}{Lemma}
  \newtheorem{prop}[thm]{Proposition} \newtheorem*{prop*}{Proposition}
\theoremstyle{definition}
  \newtheorem{defn}[thm]{Definition} \newtheorem*{defn*}{Definition}
  \newtheorem{ex}[thm]{Example}      \newtheorem*{ex*}{Example}
\theoremstyle{remark}
  \newtheorem{rk}[thm]{Remark}  \newtheorem*{rk*}{Remark}
\newcommand{\Case}[1]{\smallskip \textbf{Case #1:}}
\newenvironment{claimproof}
  {
    \begin{proof}[Proof of claim]
    \renewcommand{\qedsymbol}{\ensuremath{\circ}}
  } {
    \end{proof}
  }

% custom commands
\DeclareMathOperator{\Cg}{Cg}
\DeclareMathOperator{\Clo}{Clo}
\DeclareMathOperator{\Con}{Con}
\DeclareMathOperator{\Inv}{Rel}
\DeclareMathOperator{\Pol}{Pol}
\DeclareMathOperator{\Sg}{Sg}
\DeclareMathOperator{\diag}{diag}
\newcommand{\bmat}[1]{ \begin{bmatrix} #1 \end{bmatrix} }
\newcommand{\Bmat}[1]{ \begin{Bmatrix} #1 \end{Bmatrix} }
\newcommand{\pmat}[1]{ \begin{pmatrix} #1 \end{pmatrix} }
\newcommand{\mat}[1]{ \begin{matrix} #1 \end{matrix} }
\newcommand{\ds}[1]{ \displaystyle{#1} }
\newcommand{\stack}[2]{\genfrac{}{}{0pt}{}{#1}{#2}}
\newcommand{\Proofitem}[1]{\smallskip \textbf{#1:}}

% tikz
\usetikzlibrary{calc, matrix, positioning}
\newcommand{\TikzCenterNode}{\node (center) at ($(current bounding box.north east)!0.5!(current bounding box.south west)$) {\phantom{$\cdot$}};}

% misc
\numberwithin{equation}{section}  % number equations within sections
\renewcommand{\epsilon}{\varepsilon}
\renewcommand{\phi}{\varphi}
\newcommand{\todo}[1]{\noindent\textbf{TODO: #1}}
\renewcommand*{\arraystretch}{1.25} 
\setlist[enumerate]{itemsep=0.5em}
\setlist[itemize]{itemsep=0.5em}

% homework/exam specific
\usepackage[margin=1.5in, top=1in, bottom=1in]{geometry}
\usepackage{environ}
\newlist{questions}{enumerate}{3}
\setlist[questions]{wide=0pt, parsep=0.5em, listparindent=\parindent, font=\bfseries}
\setlist[questions,1]{label=\arabic*., itemsep=2\parsep, topsep=\itemsep+\parsep}
\setlist[questions,2]{labelindent=\parindent, label=(\roman*), itemsep=\parsep}
\setlist[questions,3]{labelindent=2\parindent, label=(\alph*)}
\newcommand{\SolnItem}[1]{\smallskip\indent\textrm{\bf#1:}}
\newenvironment{proofpart}[1] { \begin{proof}[\SolnItem{#1} Proof.] } { \end{proof} }
\newif\ifsolutions
\NewEnviron{solution}
  {
    \ifsolutions
      \bigskip \noindent \textbf{Solution:} \BODY 
    \fi
  }
  { }

% document specific stuff
\newcommand{\Prob}{\mathcal{P}}

% quantum
\DeclareMathOperator{\Cspan}{ \CC-span }
\DeclareMathOperator{\Tr}{ Tr }
\newcommand{\B}{ \mathfrak{B} }
\newcommand{\bra}[1]{ \left< #1 \right| }
\newcommand{\ket}[1]{ \left| #1 \right> }
\newcommand{\bracket}[2]{ \left< #1 \mid #2 \right> }
\tikzset{gate-c/.style = {draw, circle, inner sep=0.25em}}
\tikzset{gate-r/.style = {draw, rounded corners, rectangle}}
\tikzset{control/.style = {draw, fill, circle, inner sep=0.1em}}
\tikzset{circuit/.style = {matrix of math nodes, cells={anchor=center}, column 1/.style={anchor=east}}}
\newcommand{\tikzMeasure}{
  \begin{tikzpicture}[scale=0.2, baseline=0pt]
    \draw (0,0) arc (180-20:20:1);
    \draw[->] (current bounding box.north|-current bounding box.south) -- ++(60:1.5);
  \end{tikzpicture} }

% computation
%\usepackage[final]{minted}
%\usemintedstyle{bw, mathescape, escapeinside=;;}
\renewcommand{\O}{\mathcal{O}}
\DeclareMathOperator{\poly}{poly}
%----------------------------------------------------------------------------}}}1

\begin{document}  %\solutionstrue
% title matter {{{1
\title[Quantum Algorithms, HW 9 Additional Problems]
  {Quantum Algorithms \\ Homework 9 Additional Problems}
\author{Prof.~Matthew Moore}
\maketitle

\vspace{-0.7em} \begin{center}
  \sc Due: 2021-04-13
\end{center}
%----------------------------------------------------------------------------}}}1

\vspace{2em}

% subspace orthogonality defn   {{{1
% ID: Quoo6kae
\begin{defn*}
Let $\alg{V}$ be a vector space and let $\alg{A}, \alg{B}\leq \alg{V}$ be
subspaces.
\begin{itemize}
  \item We say that $\alg{A}$ is \emph{orthogonal} to $\alg{B}$ if for every
    $\ket{a}\in A$ and $\ket{b}\in B$ we have $\bracket{a}{b} = 0$.
  \item Define the \emph{sum} of $\alg{A}$ and $\alg{B}$ to be
    $\ds{
      \alg{A} + \alg{B}
      = \Big\{ \ket{a} + \ket{b} \mid \ket{a}\in A, \ket{b}\in B \Big\}.
    }$
\end{itemize}
\end{defn*}
%----------------------------------------------------------------------------}}}1
\begin{questions}
% quantum probability, orthogonality   {{{1
% ID: Eiph6eel
\item Suppose that $\alg{A}$ and $\alg{B}$ are orthogonal to each other.
\begin{questions}
  \item What is $\dim(\alg{A}+\alg{B})$?
  \item Show that $\Prob(\ket{v}, \alg{A}+\alg{B}) = \Prob(\ket{v}, \alg{A}) +
    \Prob(\ket{v}, \alg{B})$.
  \item Show that $\Pi_{\alg{A}}\Pi_{\alg{B}} = \Pi_{\alg{B}}\Pi_{\alg{A}}$.
\end{questions}
%----------------------------------------------------------------------------}}}1
% quantum probability, tensors   {{{1
% ID: huash2oJ
\item Suppose that $\alg{A}\leq \alg{V}$ and $\alg{B}\leq \alg{W}$ are two
subspaces. 
\begin{questions}
  \item Prove that $\Pi_{\alg{A}\otimes \alg{B}} = \Pi_{\alg{A}}\otimes
    \Pi_{\alg{B}}$.
  \item Let $\rho$ and $\tau$ be density matrices. Prove that
    $\Prob(\rho\otimes\tau, \alg{A}\otimes \alg{B}) =
    \Prob(\rho,\alg{A})\Prob(\tau,\alg{B})$. You may use the fact that
    $\Tr(X\otimes Y) = \Tr(X)\Tr(Y)$.
\end{questions}
%----------------------------------------------------------------------------}}}1
\end{questions} \end{document}
