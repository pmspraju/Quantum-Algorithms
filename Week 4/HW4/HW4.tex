\documentclass [12pt]{article}
\setlength{\parindent}{0em}
\setlength{\parskip}{0.25in}
\usepackage{geometry}
\geometry{verbose,letterpaper,tmargin=0.5in,bmargin=1.0in,lmargin=.70in,rmargin=.70in}
\usepackage{graphicx}
\usepackage{amsmath}
\usepackage{amssymb}
\usepackage{amsthm}
\theoremstyle{definition}
\newtheorem{exmp}{Example}[section]
\usepackage{tikz}
\usetikzlibrary{arrows,decorations.pathmorphing,backgrounds,positioning,fit,petri,calc,matrix}
\usepackage{slashbox}
\usepackage{listings}
\usepackage{ dsfont }
\usepackage{ upgreek }

\definecolor{dkgreen}{rgb}{0,0.6,0}
\definecolor{gray}{rgb}{0.5,0.5,0.5}
\definecolor{mauve}{rgb}{0.58,0,0.82}

\lstset{frame=tb,
  language=Python,
  aboveskip=3mm,
  belowskip=3mm,
  showstringspaces=false,
  columns=flexible,
  basicstyle={\small\ttfamily},
  numbers=none,
  numberstyle=\tiny\color{gray},
  keywordstyle=\color{blue},
  commentstyle=\color{dkgreen},
  stringstyle=\color{mauve},
  breaklines=true,
  breakatwhitespace=true,
  tabsize=3
}


\title{Home Work 3}
\author{Madhu Peduri}
\date{01/13/2021}

\begin{document}
\section*{Homework 3}

{\bf 1.} For each of the following values of q, generate 5 random members and run the Miller-Rabin test using them. What is the probability that q is prime?

\phantom{1em} {\bf q=10601}

\phantom{1em} {\bf Step 1.} $q \not\equiv 0 (mod 2)$. So q is odd. \\

\phantom{1em} {\bf Step 2.} $ (q - 1) = 2^{k}l  \Longrightarrow 10600 = 2^{3}*1325 $.\\
\phantom{1000em} k = 3; l = 1325.

\phantom{1em} {\bf Step 3.} Choose a random base $a \in \{1, 2, \dotsb 10600 \}$. \\

\phantom{1em} {\bf Step 4.} Compute the below sequence. \\
\phantom{1000em} $\{a^{l}, a^{2l}, a^{4l}, a^{8l}\}$, since k = 3.

\begin{table}[h!]
  \begin{center}
    \caption{Test for each random base.}
    \label{tab:Miller-Rabin}
    \begin{tabular}{c|l|c|c} % <-- Alignments: 1st column middle, 2nd left, 3rd middle and 4th middle, with vertical lines in between
      \textbf{Base} & \textbf{Sequence} & \textbf{Test 1} & \textbf{Test 2}\\
      \hline
      192 & 2892, 10076, 10600, 1 & Prime & Composite\\
      7219 & 7709, 10076, 10600, 1 & Prime & Composit\\
      5435 & 8244, 525, 10600, 1 & Prime & Composite\\
      1169 & 10076, 10600, 1, 1 & Prime & Compoiste\\
      16 & 1, 1, 1, 1 & Prime & Prime\\
    \end{tabular}
  \end{center}
\end{table}

\phantom{1em} {\bf Result.} q = 10601 is a Prime and tested as prime by Miller-Rabin with Probability = 1 for the given random bases. \\\

\phantom{1em} {\bf q=101101}

\phantom{1em} {\bf Step 1.} $q \not\equiv 0 (mod 2)$. So q is odd. \\

\phantom{1em} {\bf Step 2.} $ (q - 1) = 2^{k}l  \Longrightarrow 10600 = 2^{2}*25275 $.\\
\phantom{1000em} k = 2; l = 25275.

\phantom{1em} {\bf Step 3.} Choose a random base $a \in \{1, 2, \dotsb 101100 \}$. \\
\newpage
\phantom{1em} {\bf Step 4.} Compute the below sequence. \\
\phantom{1000em} $\{a^{l}, a^{2l}, a^{4l}\}$, since k = 2.

\begin{table}[h!]
  \begin{center}
    \caption{Test for each random base.}
    \label{tab:Miller-Rabin}
    \begin{tabular}{c|l|c|c} % <-- Alignments: 1st column middle, 2nd left, 3rd middle and 4th middle, with vertical lines in between
      \textbf{Base} & \textbf{Sequence} & \textbf{Test 1} & \textbf{Test 2}\\
      \hline
       21082& 39885, 90091, 1 & Prime & Composite\\
       101046 & 42834, 71709, 82720 & Composite & Prime\\
       92196 & 16666, 31109, 31109 & Composite & Prime\\
       72167 & 24452, 90091, 1 & Prime & Composite\\
       47752 & 86659, 1, 1 & Prime & Composite\
    \end{tabular}
  \end{center}
\end{table}

\phantom{1em} {\bf Result.} q, $101101 \equiv 0 (mod 7)$, is a Composite and tested by Miller-Rabin as prime with Probability = 1 for the given random bases. \\

\phantom{1em} {\bf q=15841}

\phantom{1em} {\bf Step 1.} $q \not\equiv 0 (mod 2)$. So q is odd. \\

\phantom{1em} {\bf Step 2.} $ (q - 1) = 2^{k}l  \Longrightarrow 10600 = 2^{5}*495$.\\
\phantom{1000em} k = 5; l = 495.

\phantom{1em} {\bf Step 3.} Choose a random base $a \in \{1, 2, \dotsb 15840 \}$. \\
 
\phantom{1em} {\bf Step 4.} Compute the below sequence. \\
\phantom{1000em} $\{a^{l}, a^{2l}, a^{4l}, a^{8l}, a^{16l}, a^{32l}\}$, since k = 5.

\begin{table}[h!]
  \begin{center}
    \caption{Test for each random base.}
    \label{tab:Miller-Rabin}
    \begin{tabular}{c|l|c|c} % <-- Alignments: 1st column middle, 2nd left, 3rd middle and 4th middle, with vertical lines in between
      \textbf{Base} & \textbf{Sequence} & \textbf{Test 1} & \textbf{Test 2}\\
      \hline
       14293& 6852, 13021, 218, 1, 1, 1 & Prime & Composite\\
       15346 & 1, 1, 1, 1, 1, 1 & Prime & Prime\\
       2472 & 12461, 3039, 218, 1, 1, 1 & Prime & Composite\\
       2698 & 776, 218, 1, 1, 1, 1 & Prime & Composite\\
       5057 & 3380, 3039, 218, 1, 1, 1 & Prime & Composite\
    \end{tabular}
  \end{center}
\end{table}

\phantom{1em} {\bf Result.} q, $15841 \equiv 0 (mod 7)$, is a Composite and tested by Miller-Rabin as Prime with Probability = 1 for the given random bases. \\

\newpage

{\bf 2.} Compute the following

\phantom{1em} {\bf 1. $7^{7}$  in $ \mathds {Z}4$ }

\phantom{1em} In $ \mathds {Z}4, [7^{7}] =  7^{7} $ mod 4 \\
\phantom{1em} In $ \mathds {Z}4 , [7] = [3] $ and $7^{3}mod 4 = 3$\\
\phantom{1em} We can write $7^{7} = 7^{3} * 7^{3} * 7  \Longrightarrow 7^{7} mod 4 = (7^{3} mod 4) * (7^{3}mod 4) *( 7 mod 4) mod 4$\\
\phantom{1em} By substitution, we get, $7^{7} mod 4 = 27 mod 4 = 3 \ in \ \mathds {Z}4$ 

\phantom{1em} {\bf 2. $7^{7^{7}}$  in $ \mathds {Z}4$ }

\phantom{1em} In $ \mathds {Z}4, [7^{7^{7}}] =  7^{7^{7}} $ mod 4 \\
\phantom{1em} From above problem we know, $7^{7}$ = 3 in $ \mathds {Z}4$ \\
\phantom{1em} $7^{7^{7}} \ mod 4 = ((7^{7} mod 4)^{7}) mod 4$\\
\phantom{1em} $7^{7^{7}} \ mod 4 = 3^{7} mod 4 = 3$\\
\phantom{1em} $7^{7^{7}} = 3 \ in \ \mathds {Z}4$ 

\phantom{1em} {\bf 3. $7^{7^{7^{7}}}$  in $ \mathds {Z}5$ }

\phantom{1em} In $ \mathds {Z}5, [7^{7^{7^{7}}}] = 7^{7^{7^{7}}} $ mod 5 \\
\phantom{1em} 5 is a prime number and is a coprime to 7, so by Fermat's little therom, $7^{4}mod5 = 1$\\
\phantom{1em} Let say $7^{7^{7}} = r + 4k$, then by applying Fermat's therom, $7^{7^{7^{7}}}  mod 5 = (7^{r} mod5)mod5$\\
\phantom{1em} If  $7^{7^{7}} = r + 4k$, then $r = 7^{7^{7}} mod 4$\\
\phantom{1em} By above problem, we know that $7^{7^{7}} mod 4 = 3 \Longrightarrow r = 3$\\
\phantom{1em} By substitution, $7^{7^{7^{7}}} mod 5 =  (7^{3} mod5)mod5 = 3mod5$\\
\phantom{1em} In  $ \mathds {Z}5, 3mod5 = 3 \Longrightarrow [7^{7^{7^{7}}}] = 3$

{\bf 3.} Compute $2^{3^{4^{5}}} mod \ 79$

\phantom{1em} 79 is a prime number and 2,7 are coprime. By Fermat's little therom, $2^{78} mod 79 = 1$\\
\phantom{1em} Let $3^{4^{5}} = r + 78k$, then, $2^{3^{4^{5}}} mod \ 79 = (2^{r})(2^{78})^{k}mod79$\\
\phantom{1em} By applying Fermats therom, $2^{3^{4^{5}}} mod \ 79 = (2^{r})mod79$ \\
\phantom{1em} If $3^{4^{5}} = r + 78k \Longrightarrow r =  3^{4^{5}} mod 78$ \\
\phantom{1em} We can factorize $78 = 2 * 3* 13$. Now we find modulus of $3^{4^{5}}$ for each factor \\ 
\phantom{1em} $3^{4^{5}} mod 2 = 1 mod 2$ \\
\phantom{1em} $3^{4^{5}} mod 3 = 0 mod 3$ \\
\phantom{1em} For $3^{4^{5}} mod 13$, let $4^{5} = x + 12s$, then $3^{4^{5}} mod 13 = (3^{4})(3^{12})^{s}mod13$\\
\phantom{1em} 13 is a prime and 3, 13 are coprime, by applying Fermats therom, \\
\phantom{1em} we can say, $3^{4^{5}} mod 13 = 3 mod 13$, where x=4\\
\phantom{1em} Now we have 3 modulo, $y = 1 mod 2; y = 0 mod 3; y= 3 mod 13$\\
\phantom{1em} If we apply chinese remainder therom to solve above modulo, we get $3^{4^{5}} mod 78 = 237 mod 78$\\
\phantom{1em} $\Longrightarrow r = 237 mod 78 = 3 \Longrightarrow  2^{3^{4^{5}}} mod \ 79 = (2^{3})mod79$\\

\newpage

{\bf 4.} Prove that if $gcd(m,n) = 1 \ then \ \upvarphi (m.n) = \upvarphi(m) . \upvarphi (n)$

\phantom{1em} Given function $ \upvarphi (n) $ is a set of integers obtained by a modulo (n).\\ 
\phantom{1em} It was given that $gcd(m,n) = 1 \Longrightarrow integer \ m, \ n \ are \ coprime$. 

\phantom{1em} Chinese remainder theorem says that if $q = b_{1}.b_{2}$ where  $b_{1} , b_{2}$ \\
\phantom{1em} are positive integers and such that $gcd(b_{1} , b_{2}) = 1$, then the below map is isomorphic,\\
\phantom{1em} that is one-to-one, so, $ \lambda _{q,(b_{1} , b_{2})} = \mathds {Z} /q\mathds {Z} = (\mathds {Z}/b_{1} \mathds {Z}) \times (\mathds {Z}/b_{2} \mathds {Z})$

\phantom{1em} If we substitiute $b_{1} = m, b_{2} = n$ we get below mapping.\\
\phantom{1em} $(\mathds {Z} /mn\mathds {Z})^{\times} = (\mathds {Z}/m \mathds {Z})^{\times} \times (\mathds {Z}/n \mathds {Z})^{\times}$\\
\phantom{1em} $\Longrightarrow  \upvarphi (m.n) = \upvarphi(m) . \upvarphi (n)$ 

\end{document}