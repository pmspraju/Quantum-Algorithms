\documentclass[oneside]{amsart}  % {{{1
\usepackage{amssymb, mathtools, enumitem, booktabs, float, caption, tikz}
\usepackage[final]{hyperref, microtype}

% structure symbols
\newcommand{\NN}{\mathbb{N}}
\newcommand{\ZZ}{\mathbb{Z}}
\newcommand{\QQ}{\mathbb{Q}}
\newcommand{\RR}{\mathbb{R}}
\newcommand{\CC}{\mathbb{C}}
\renewcommand{\AA}{\mathbb{A}}
\newcommand{\V}{\mathcal{V}}
\newcommand{\m}[1]{\mathbb{#1}}      % models
\newcommand{\alg}[1]{\m{#1}}         % algebras
\newcommand{\rel}[1]{\mathbf{#1}}    % relational structures
\newcommand{\cl}[1]{\mathcal{#1}}    % classes
\newcommand{\Comp}[1]{\textsf{#1}}   % complexity classes

% theorems and similar environments
\theoremstyle{plain}
  \newtheorem{thm}{Theorem}           \newtheorem*{thm*}{Theorem}
  \newtheorem{claim}{Claim}[thm]      \newtheorem*{claim*}{Claim}
  \newtheorem{conj}[thm]{Conjecture}  \newtheorem*{conj*}{Conjecture}
  \newtheorem{cor}[thm]{Corollary}    \newtheorem*{cor*}{Corollary}
  \newtheorem{lem}[thm]{Lemma}        \newtheorem*{lem*}{Lemma}
  \newtheorem{prop}[thm]{Proposition} \newtheorem*{prop*}{Proposition}
\theoremstyle{definition}
  \newtheorem{defn}[thm]{Definition} \newtheorem*{defn*}{Definition}
  \newtheorem{ex}[thm]{Example}      \newtheorem*{ex*}{Example}
\theoremstyle{remark}
  \newtheorem{rk}[thm]{Remark}  \newtheorem*{rk*}{Remark}
\newcommand{\Case}[1]{\smallskip \textbf{Case #1:}}
\newenvironment{claimproof}
  {
    \begin{proof}[Proof of claim]
    \renewcommand{\qedsymbol}{\ensuremath{\circ}}
  } {
    \end{proof}
  }

% custom commands
\DeclareMathOperator{\Cg}{Cg}
\DeclareMathOperator{\Clo}{Clo}
\DeclareMathOperator{\Con}{Con}
\DeclareMathOperator{\Inv}{Rel}
\DeclareMathOperator{\Pol}{Pol}
\DeclareMathOperator{\Sg}{Sg}
\DeclareMathOperator{\diag}{diag}
\newcommand{\bmat}[1]{ \begin{bmatrix} #1 \end{bmatrix} }
\newcommand{\Bmat}[1]{ \begin{Bmatrix} #1 \end{Bmatrix} }
\newcommand{\pmat}[1]{ \begin{pmatrix} #1 \end{pmatrix} }
\newcommand{\mat}[1]{ \begin{matrix} #1 \end{matrix} }
\newcommand{\ds}[1]{ \displaystyle{#1} }
\newcommand{\stack}[2]{\genfrac{}{}{0pt}{}{#1}{#2}}
\newcommand{\Proofitem}[1]{\smallskip \textbf{#1:}}

% tikz
\usetikzlibrary{calc, matrix, positioning}
\newcommand{\TikzCenterNode}{\node (center) at ($(current bounding box.north east)!0.5!(current bounding box.south west)$) {\phantom{$\cdot$}};}

% misc
\numberwithin{equation}{section}  % number equations within sections
\renewcommand{\epsilon}{\varepsilon}
\renewcommand{\phi}{\varphi}
\newcommand{\todo}[1]{\noindent\textbf{TODO: #1}}
\renewcommand*{\arraystretch}{1.25} 
\setlist[enumerate]{itemsep=0.5em}
\setlist[itemize]{itemsep=0.5em}

% homework/exam specific
\usepackage[margin=1.5in, top=1in, bottom=1in]{geometry}
\usepackage{environ}
\newlist{questions}{enumerate}{3}
\setlist[questions]{wide=0pt, parsep=0.5em, listparindent=\parindent, font=\bfseries}
\setlist[questions,1]{label=\arabic*., itemsep=2\parsep, topsep=\itemsep+\parsep}
\setlist[questions,2]{labelindent=\parindent, label=(\roman*), itemsep=\parsep}
\setlist[questions,3]{labelindent=2\parindent, label=(\alph*)}
\newcommand{\SolnItem}[1]{\smallskip\indent\textrm{\bf#1:}}
\newenvironment{proofpart}[1] { \begin{proof}[\SolnItem{#1} Proof.] } { \end{proof} }
\newif\ifsolutions
\NewEnviron{solution}
  {
    \ifsolutions
      \bigskip \noindent \textbf{Solution:} \BODY 
    \fi
  }
  { }

% document specific stuff
% quantum
\DeclareMathOperator{\Cspan}{ \CC-span }
\DeclareMathOperator{\Tr}{ Tr }
\newcommand{\B}{ \mathfrak{B} }
\newcommand{\bra}[1]{ \left< #1 \right| }
\newcommand{\ket}[1]{ \left| #1 \right> }
\newcommand{\bracket}[2]{ \left< #1 \mid #2 \right> }
\tikzset{gate-c/.style = {draw, circle, inner sep=0.25em}}
\tikzset{gate-r/.style = {draw, rounded corners, rectangle}}
\tikzset{control/.style = {draw, fill, circle, inner sep=0.1em}}
\tikzset{circuit/.style = {matrix of math nodes, cells={anchor=center}, column 1/.style={anchor=east}} }
\newcommand{\tikzMeasure}{
  \begin{tikzpicture}[scale=0.2, baseline=0pt]
    \draw (0,0) arc (180-20:20:1);
    \draw[->] (current bounding box.north|-current bounding box.south) -- ++(60:1.5);
  \end{tikzpicture} }

% computation
%\usepackage[final]{minted}
%\usemintedstyle{bw, mathescape, escapeinside=;;}
\renewcommand{\O}{\mathcal{O}}
\DeclareMathOperator{\poly}{poly}
%----------------------------------------------------------------------------}}}1

\begin{document}  %\solutionstrue
% title matter {{{1
\title[Quantum Algorithms, HW 10 Additional Problems]
  {Quantum Algorithms \\ Homework 10 Additional Problems}
\author{Prof.~Matthew Moore}
\maketitle

\vspace{-0.7em} \begin{center}
  \sc Due: 2021-04-20
\end{center}
%----------------------------------------------------------------------------}}}1

\vspace{2em}

\noindent
% orthogonal subgroup defn   {{{1
For a subgroup $\alg{A}\leq \ZZ_2^n$ define
\[
  \alg{A}^{\perp}
  = \big\{ g\in \ZZ_2^n \mid g\cdot a = 0 \text{ for all } a\in A \big\},
\]
where $a\cdot g$ is the dot product modulo 2 of $g$ and $g$ (regarded as
$\ZZ_2$-vectors).
%----------------------------------------------------------------------------}}}1

\bigskip

\begin{questions}
% Simon's algorithm, group theory {{{1
% ID: bahg7ooC
\item Let $\AA\leq \ZZ_2^n$ and $g\in \ZZ_2^n$. Define
\[
  A_0
  = \big\{ a\in A \mid a\cdot g = 0 \big\}
  \qquad \text{and} \qquad
  A_1
  = \big\{ a\in A \mid a\cdot g = 1 \big\}.
\]

\begin{questions}
  \item Prove that $A = A_0\cup A_1$ and $\emptyset = A_0\cap A_1$.

  \item Suppose that $a\in A_1$. Prove that $a+A_0 = A_1$ and $a+A_1 = A_0$.
  Explain why this implies $|A_0| = |A_1|$ if $A_1 \neq\emptyset$.

  \item Prove that
  \[
    \sum_{a\in A} (-1)^{a\cdot g}
    = \begin{cases}
      |A| & \text{if } a\cdot g = 0 \text{ for all } a\in A, \\
      0   & \text{otherwise}.
    \end{cases}
  \]
\end{questions}
%----------------------------------------------------------------------------}}}1
% Simon's algorithm {{{1
% ID: Aetho3he
\item Using the previous question, prove the assertion on page 119 that
\[
  \sum_{\stack{a,b}{x-y\in D}} (-1)^{a\cdot x - b\cdot y} \neq 0
\]
if and only if $a=b\in D^{\perp}$ (the book uses $E^*$).
%----------------------------------------------------------------------------}}}1
% characters   {{{1
% ID: Theekie1
\item We say that a subgroup $\AA\leq \ZZ_2^n$ is \emph{maximal} if
\begin{itemize}
  \item $A \neq \ZZ_2^n$ and
  \item if $\AA \leq \alg{X}\leq \ZZ_2^n$ then $\AA = \alg{X}$ or $\alg{X} =
    \ZZ_2^n$.
\end{itemize}
Similarly, $\AA\leq \ZZ_2^n$ is \emph{minimal} if
\begin{itemize}
  \item $\{0\} \neq A$ and
  \item if $\{0\} \leq \alg{X}\leq \AA$ then $\{0\} = \alg{X}$ or $\alg{X} =
    \AA$.
\end{itemize}
Prove that $\AA$ is maximal if and only if $\AA^{\perp}$ is minimal (use this in
your solution to 13.1).
%----------------------------------------------------------------------------}}}1
% Simon's algorithm   {{{1
% ID: eeYiu5xi
\item In Simon's algorithm, what would happen if instead of measuring the first
block of qubits, we measured the second block of qubits? Calculate the density
matrix and describe what distribution it represents.
%----------------------------------------------------------------------------}}}1
\end{questions} \end{document}
